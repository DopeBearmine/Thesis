
\documentclass{article}
\usepackage[backend=bibtex,style=authoryear,natbib=true]{biblatex} % Use the bibtex backend with the authoryear citation style (which resembles APA)
\addbibresource{library.bib} % The filename of the bibliography

\begin{document}
	Key phrases from papers on neural correlations.
	\begin{enumerate}
		\item[\parencite{Moreno-Bote2014a}] We found that large networks receiving finite information must contain correlations approximately proportional to the product of the derivatives of the tuning curves (referred to as differential correlations, see below), which are solely responsible for the information limitation.
		\item[\parencite{Moreno-Bote2014a}] Thus, only differential correlations can make information saturate as N increases; other correlations can decrease information, but cannot make it saturate.
		\item[\parencite{Moreno-Bote2014a}] Notably, the presence of the differential component cannot be revealed by plotting the correlations as a function of the difference in preferred stimuli. The correlation coefficients estimated empirically from 1,000 trials looked essentially the same whether or not there were information-limiting correlations (Fig. 6c).
		\item[\parencite{Moreno-Bote2014a}] First, we found that, when information is limited, the limit is a result of differential correlations; that is, correlations proportional to the product of the derivatives of the tuning curves.
	\end{enumerate}
\end{document}