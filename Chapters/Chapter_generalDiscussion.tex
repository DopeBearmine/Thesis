\chapter{\color{thesisBlue} General Discussion} % Main chapter title

\label{ch:generalDiscussion} % Change X to a consecutive number; for referencing this chapter elsewhere, use \ref{ChapterX}
\glsresetall

%----------------------------------------------------------------------------------------
%	SECTION 1
%----------------------------------------------------------------------------------------

In this thesis, I have proposed and executed a series of studies investigating how neurons encode stimulus information across a population. 

The first study (Chapter \ref{ch:papernak}), looked at how feature tuning in the \gls{mt} can change at an individual and population level, depending on behavioral context. I showed that the amount of stimulus information conveyed by \gls{mt} neurons changes depending on the animals goals. Counter-intuitively, I found that a substantial population of neurons in area \gls{mt} actually decrease the amount of conveyed stimulus information when those stimuli are relevant to the animal. While these neurons participated less in the sensory signaling aspect of the task, they also contributed more to the sensory decision making process. This division of labor demonstrates yet again the flexibility and efficiency of the brain in solving sensory problems. This same concept of flexible and efficient sensory coding motivated the second study in this thesis, Chapter \ref{ch:maps}.

We sought to maximize the information gained about neural sensory-response functions using a single experiment. To this end we leveraged the rich, continuous, and naturalistic feature space of a \gls{gan} with optimization techniques such as \gls{pso} on large neural ensembles. By optimizing on the $L^2$ norm of the population response vector, we found that both \gls{pso} and genetic algorithms improved the quality of stimulus-response information gained in the experiment defined by stronger linear and nonlinear relationships present between neural responses and \gls{gan} latents. We also found striking similarities across algorithms in the types of features found to be relevant to the neurons, including color palettes, textures and shapes. Interestingly, \gls{gan} latent variables proved to be better at explaining neurons' behavior than even the images themselves at the individual-neuron level. This suggests that the brain may in fact be using some form of latent variable model for representing visual information. So for example, color, may not be relevant to the brain in and of itself, but a nonlinear combination of color, contrast, spatial orientation, etc is a more effective or efficient way to process visual information. As artificial neural networks such as \gls{gan}s take inspiration from the brain, this follows naturally. Another striking finding from this experiment is the degree to which V1 and V4 respond linearly. We found that despite the complexity of stimuli presented in this study V1 still maintained a linear stimulus embedding, while V4 spanned both linear and nonlinear stimulus-response functions. This result held regardless of predictor, in that pixel values and \gls{gan} latent variables had similar trends. 

Taken together, these studies form the foundation of an approach to sensory coding that takes into account factors that are not typically accounted for in experimental designs by necessity. By accounting for behavioral context (active vs. passive tasks) we demonstrated that the \gls{mt} alters sensory processing in many (even detrimental) ways, but that these effects served to aid other parts of the task. Then, by accounting for untested feature dimensions, we demonstrated that neurons are tuned to complex, nonlinear combinations of features, and that these relationships can be parsed out with the use of artificial intelligence. Future work into the relationship between visual stimuli and neural responses should combine the studies to investigate how high dimensional neural manifold geometry changes with task design. Limitations in the first study (few simultaneously recorded neurons) and the second (no behavioral task) could be removed by investigating how neural populations encode complex feature sets, and how that encoding changes according to subject goals. 