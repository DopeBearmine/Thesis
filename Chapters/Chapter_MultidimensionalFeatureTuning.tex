% Chapter 

\chapter{Neural Manafold Approximation with Particle Swarm (nMAPS)} % Main chapter title

\label{ch:maps} % For referencing the chapter elsewhere, use \ref{ch:maps} 

%----------------------------------------------------------------------------------------

% Define some commands to keep the formatting separated from the content 
%\newcommand{\keyword}[1]{\textbf{#1}}

%------------------------------------------------------------------------------------

\section{Introduction}
In the previous section I discussed how optimization approaches can circumvent some of the experimental limitations for measuring high-dimensional feature tuning. This was done on model neural populations with a known ground truth. In this section I will discuss similar methodology using electrophysiology recordings in awake and behaving Rhesus Macaques.

Research into sensory coding in the visual system focuses on determining what visual features neuronal activity covaries with, i.e. what information are they encoding. Traditional experiments looked at neural responses to simple stimuli such as small patches of light or oriented bars \cite{Hubel1959}. These stimuli are effective at driving neurons in early visual areas, but produce Weak responses from higher visual areas where neurons prefer complex combinations of many features like contrast, color, orientation, curvature, etc \cite{Sani2013,Tanigawa2010, Nandy2016}. In these higher visual areas, like V4, it is not feasible to systematically test every combination of visual features to search for feature tuning. Here I describe a method for efficiently exploring these spaces to find relationships between sensory features and neural responses. 

\section{\color{RoyalBlue!50!black} Methods}
\subsection{Research subject}
One adult male rhesus macaque (\textit{Macaca mulatta}) was used for this study.
All experimental procedures were approved by the University Committee on Animal Resources at the University of Rochester and performed in accordance with the National Institutes of Health \textit{Guide for the Care and Use of Laboratory Animals} \cite{AnimalUse2011}.

\subsection{Generative Adversarial Networks}
\glsreset{gan}
\Glspl{gan} are a type of artificial neural network that learn low-dimensional representations of higher-dimensional data, such as images \cite{Karras2019}. \glspl{gan} trained to generate images learned to map high-order image statistics onto a set of `latent' variables.  \glspl{gan} provide a continuous stimulus space that is constrained to be within a set of plausibly natural images. The images generated by a \gls{gan} are more naturalistic than simple Gabors, while also being more tractable to use experimentally than natural images, making them promising tools for vision research. The \gls{gan} used in these experiments had a 128-dimensional input (latent) space, and was trained on the Cifar-10 image data set as described previously \cite{Fruend2018}. 

\subsection{Particle Swarm}
Particle swarm utilizes a hive-mind approach to solve optimization problems. Each `particle’ corresponds to a point in the high dimensional stimulus space (an image), and travels through the space in order to maximize the $L^2$ norm of the population response. By moving through the stimulus space, the images change along latent feature dimensions, resulting in smooth image manipulation. Particles move through the stimulus space by integrating information about local gradients and neural responses to other particles. Particles therefore both compete and collaborate in order to explore the stimulus and response spaces. 

\gls{maps} is initialized with three generations of random points for each of 64 particles. This gives the kernel regression a history to estimate gradients. For subsequent generations, each particle takes a step $S_p$ according to three terms: the estimated local gradient $\nabla e_p$,  the weighted sum of vectors towards particles that resulted in better neural responses $G_p$, and a momentum term $M_p$ (Equation 1).
\begin{equation}
S_p= c_1 r_{1p} \nabla e_p+ c_2  r_{2p} G_p+b M_p
\end{equation}

The constants $c_1$ and $c_2$ are learning rates for the gradient and global information components respectively, while $b \in [0,1]$ is a decay term for the momentum. The stochastic scaling factors $r_{1p}$ and $r_{2p} \sim U(0,1)$ help circumvent the problem of choosing a correct step size. Too large and particles will jump over maxima, but too small and it will take too long to converge, so step sizes are pulled from a uniform distribution for each particle each generation.

We used kernel regression to estimate the gradient according to each particles’ personal history. This way, each particle travels along its' local gradient, independent of how each other particle moves. 

\begin{equation}
\nabla e_p (x^* )=\frac{1}{t-1} \sum_{k=1}^{t-1} \nabla w_k (x^* )  A_k
\end{equation}

\begin{equation}
\nabla w_k (x^* )=\frac{2K(x^*,x_k) \sum_{l=1}^{t-1} (x_k-x_l)K(x^*,x_l)} {h^2\left(\sum_{l=1}^{t-1}K(x^*, x_l)\right)^2}
\end{equation}

where

\begin{equation}
A_k = \sum_{j=1}^{t-1}||r_k||+||r_k-r_j||
\end{equation}

and

\begin{equation}
K(x,y)=e^{- \frac{||x-y||}{h^2}}
\end{equation}

The next component (Equation \ref{globalPart}) is a weighted sum that uses information about the global response manifold to predict where good parts of the space is. Stimuli that resulted in a large neural response from many neurons pull the particle density toward them. Again, $x_p$ and $r_p$ represent the stimulus embedding vector and the neural response vector respectively for particle $p$. $G_p$ is the sum of unit vectors towards all the points which resulted in better neural responses $x_k, k = 1,...,u$, weighted by how much better that response $r_k, k = 1,...,u$ was than the current point $r_p$.

\begin{equation}
%G_p = ||\sum_{k=1}^{u}(||r_k||-||r_p||) ||\label{globalPart}
G_p = \sum_{k=1}^{u}\frac{(||r_k||-||r_p||)}{\sum_{j=1}^{u}||r_j||-||r_p||} \frac {x_k-x_p}{||x_k-x_p||}
\label{globalPart}
\end{equation}

\subsection{Genetic Algorithm}



\subsection{Electrical Recordings}
Prior to any electrical recordings, the subject was first implanted with a titanium head holder, trained on behavioral tasks, and then implanted with chronic electrode arrays. All surgeries we performed with isoflurane anesthesia, aseptic technique, and perioperative opiate analgesics.

\textit{Head Holder Implantation}. The subject was initially sedated with a 10mg/kg intramuscular injection of ketamine and administered with 0.25mg/kg midazolam, 0.011mg/kg glycopyrrolate, 25mg/kg cefazolin, and 0.2mg/kg meloxicam, all intramuscular. Once anesthetized, we intubated the animal with an endotrachial tube, shaved the animals head, inserted a catheter in the small saphenous vein for infusion of lactated Ringer's solution, then positioned the head in a stereotaxic frame. The anesthesia was maintained with 1.5\% isoflourane throughout the surgery. The surgical site was then thoroughly scrubbed with povidone iodine solution in the preparation of the sterile field. We made a horseshoe-shaped incision (6cm wide by 10cm anterior-posterior) using a \#10 scalpel blade starting from the right brow, cutting caudally parallel to the midline, and ending at the left brow. We then used bone curettes to retract the tissue and clear a large enough surface of skull for the head holder. Sterile gauze soaked in saline was used to keep the tissue moist throughout the surgery. The prefabricated titanium head holder (custum design machined by the university of Pittsburgh cite mat?) attaches to the skull using 16 6mm titanium screws (Veterinary Orthopedic Implants, Saint Augustine, FL) that go through 6 flange straps radially extending from the center of its base. We next bent the straps of the head holder so it fit tightly on the skull. We marked with pencil the final position of the head holder on the skull and made two small incisions in an x through the retracted tissue where the head holder would protrude. We then pushed the exterior portion of the head holder through the dermostomy in the retracted tissue and repositioned it back on the skull. We used a 2mm surgical drillbit and a custum drill stop set to the thickness of the skull to drill through the skull, starting with the lateral-most location of the head holder. Skrews were implanted one a time, alterniting across the midline, lateral to medial. Once each screw was implanted we used geristore (DenMat, Lompoc, CA, USA) to fill any gaps between the skull and head holder. After the geristor fully dried, we sutured the wound using a continuous running stitch to attach the subcutaneous fascia and simple interrupted stitches on top to fully connect the skin around the wound margin. Intramuscular injections of 25mg/kg cefazolin were administered every 12h for seven days post op, and 0.2mg/kg meloxicam every 24h for 3 days post op. The subject was allowed one month to recover before the start of training.

\textit{Behavioral Training}. Prior to array implantation, the animal first underwent fixation training. The monkey sits 50cm away from a 120Hz ViewPixx/3D monitor (VPixx Technologies, Saint-Bruno, QC, Canada). We used Matlab (The MathWorks, Inc) and Psychtoolbox to control the experiments and present visual stimuli. Eye position was tracked with an Eyelink 1000 IR eye tracking camera (SR Research, Ottawa, Ontario, Canada), and the monkey was given water reward for fixating on a central dot. Once the monkey understood the basic reward contingencies he was implanted with two 128 channel matrix electrode arrays (NeuroNexus, Ann Arbor, MI, USA).

\textit{Array Implantation}. The monkey was given intramuscular injections of ketamine, medazolam, glycopyrrolate, cefazolin, and meloxicam at the same doses described earlier for the headpost holder surgery. Similarly, the monkey was intubated, shaved, catheterized, and positioned in the stereotactic frame in the same manner. Again the animal was maintained on 1.5\% anesthesia throughout the surgery. We marked the stereotactic coordinates for prefrontal cortex (30mm anterior, 21mm lateral, 25mm dorsal) and visual area V4 (0mm anterior, 0mm lateral, 27mm dorsal) as well as estimated locations for the two array pedestals (NeuroNexus, Ann Arbor, MI, USA) prior to making the incision. Once we planned out the location of the pedestals and craniotomies we made the incision with a size 10 scalpel blade, starting on the posterior surface of the cranium. The incision was made just off the midline (away from the hemisphere we implanted in) and continued until about 1cm away from the margin around the head holder, leaving room to ensure that enough healthy tissue remained between the incision and head holder. We then cut a hemicircle around the head holder, again leaving enough healthy tissue to be able to suture the wound and continued just off the midline up to the brow. The final incision started halfway through the hemicircular incision, perpendicular to the midline, and continued 10cm laterally. We used bone curettes to retract the tissue while minimizing muscle damage, and again kept the retracted tissue hydrated with sterile gauze and saline. Once the skull was sufficiently cleaned, we marked the location of the craniotomies in pencil using stereotactic coordinates. The pedestals were both placed on the midline, one anterior to the head holder (PFC), and the other posterior to the head holder (V4). Similar to the head holder surgery, we then bent the legs of the two pedestals to fit tightly onto the skull and marked the location of each screw with a pencil. The animal was then administered a second intramuscular dose of cefazolin. We used 8 and 10 6mm titanium screws (Veterinary Orthopedic Implants, Saint Augustine, FL) for the PFC and V4 pedestals respectively. After the pedestals were secured we used a 19mm diameter trephine to do the PFC craniotomy followed by kerrison punches to remove any pieces of bone around the perimeter. The animal was then administered a 0.5mg/kg intramuscular dose of dexmethasone. Afterward, we used a drimmel to smooth down a trench going from the pedestal to the craniotomy in order to prevent the wire bundle from snagging or bending. We next cut the dura on three sides of a 1cm square and retracted it. A 128 channel matrix electrode array (NeuroNexus, Ann Arbor, MI, USA) was inserted dorsal to the principal sulcus, using a microdriver (Zaber, Vancouver, British Columbia, Canada) attached to an all-angle manipulator (NeuroNexus, Ann Arbor, MI, USA). Once the array was in place, we inserted the reference wire under the dura, sutured it closed with nurolon sutures (5-0 dissolving sutures), and covered the craniotomy with duragen (Integra Life Sciences co.). The animal was injected with Buprinex SR IM every two hours, starting 6 hours into the surgery until we finished. We then used two titanium plates to cover and protect the craniotomy by screwing them into the skull in the same manner as the pedestals and headpost. After the craniotomy was secure we covered the trench and wire bundle with kwik-sil and filled any gaps under the PFC pedestal with geristore. We then sutured around the anterior wound margin back to the lateral incision using the same procedure as the head holder sutures. These same steps from craniotomy to suturing were replicated for the implantation of the V4 array. Post operative care included administration of 25mg/kg cefazolin every 12h for seven days, and 0.2mg/kg meloxicam every 24h for 3 days.
\\

\subsection{Data Analysis}
\hspace*{10mm}All data was analyzed with Matlab 2018b (The MathWorks, Inc) statistics and machine learning toolbox. A neuron's response was defined as the number of threshold crossings (-4 standard deviations below the median noise) that occurred in a 200ms window from stimulus onset. Channels with unrealistic firing rates above 300 Hz (ie. 60+ spikes within the window) had their responses set to 0 so as to not adversely affect the optimization algorithm. Additionally, if the population response on any trial resulted in too large an $L^2$ norm, this was assumed to be due to noise and the trial was repeated before the end of the block. We used canonical correlations analysis (CCA) and distance covariance analysis (DCA) to analyze relative linear and nonlinear relationships between the GAN latent space and neural response space \cite{Cowley2017b}. 



%------------------------------------------------------------------------------------

\section{High-Dimensional Feature Tuning}
\subsection{Neurons multiplex}
\subsection{How they multiplex is a mystery in V4}
\subsection{Stimulus encoding is a population-level problem}


\section{High-dimensional feature spaces (GAN)}
\subsection{WTF is a latent variable model}
\subsection{WTF is a GAN? also why would you do this to yourself?}


\section{Comparing two high-dimensional spaces}
\subsection{Population-level results} % CCA, DCA, etc
\subsection{Individual Neuron results} % Poiss Regression
\subsection{Converting "latents" back into english} % image analysis 




%------------------------------------------------------------------------------------


 